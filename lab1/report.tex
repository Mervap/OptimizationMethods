\documentclass{article}


    
\usepackage[breakable]{tcolorbox}
    \usepackage{parskip} % Stop auto-indenting (to mimic markdown behaviour)
    
    \usepackage{iftex}
    \ifPDFTeX
    	\usepackage[T1]{fontenc}
    	\usepackage{mathpazo}
    \else
    	\usepackage{fontspec}
    \fi

    % Basic figure setup, for now with no caption control since it's done
    % automatically by Pandoc (which extracts ![](path) syntax from Markdown).
    \usepackage{graphicx}
    % Maintain compatibility with old templates. Remove in nbconvert 6.0
    \let\Oldincludegraphics\includegraphics
    % Ensure that by default, figures have no caption (until we provide a
    % proper Figure object with a Caption API and a way to capture that
    % in the conversion process - todo).
    \usepackage{caption}
    \DeclareCaptionFormat{nocaption}{}
    \captionsetup{format=nocaption,aboveskip=0pt,belowskip=0pt}

    \usepackage{float}
    \floatplacement{figure}{H} % forces figures to be placed at the correct location
    \usepackage{xcolor} % Allow colors to be defined
    \usepackage{enumerate} % Needed for markdown enumerations to work
    \usepackage{geometry} % Used to adjust the document margins
    \usepackage{amsmath} % Equations
    \usepackage{amssymb} % Equations
    \usepackage{textcomp} % defines textquotesingle
    % Hack from http://tex.stackexchange.com/a/47451/13684:
    \AtBeginDocument{%
        \def\PYZsq{\textquotesingle}% Upright quotes in Pygmentized code
    }
    \usepackage{upquote} % Upright quotes for verbatim code
    \usepackage{eurosym} % defines \euro
    \usepackage[mathletters]{ucs} % Extended unicode (utf-8) support
    \usepackage{fancyvrb} % verbatim replacement that allows latex
    \usepackage{grffile} % extends the file name processing of package graphics 
                         % to support a larger range
    \makeatletter % fix for old versions of grffile with XeLaTeX
    \@ifpackagelater{grffile}{2019/11/01}
    {
      % Do nothing on new versions
    }
    {
      \def\Gread@@xetex#1{%
        \IfFileExists{"\Gin@base".bb}%
        {\Gread@eps{\Gin@base.bb}}%
        {\Gread@@xetex@aux#1}%
      }
    }
    \makeatother
    \usepackage[Export]{adjustbox} % Used to constrain images to a maximum size
    \adjustboxset{max size={0.9\linewidth}{0.9\paperheight}}

    % The hyperref package gives us a pdf with properly built
    % internal navigation ('pdf bookmarks' for the table of contents,
    % internal cross-reference links, web links for URLs, etc.)
    \usepackage{hyperref}
    % The default LaTeX title has an obnoxious amount of whitespace. By default,
    % titling removes some of it. It also provides customization options.
    \usepackage{titling}
    \usepackage{longtable} % longtable support required by pandoc >1.10
    \usepackage{booktabs}  % table support for pandoc > 1.12.2
    \usepackage[inline]{enumitem} % IRkernel/repr support (it uses the enumerate* environment)
    \usepackage[normalem]{ulem} % ulem is needed to support strikethroughs (\sout)
                                % normalem makes italics be italics, not underlines
    \usepackage{mathrsfs}
     % load all other packages
% For cyrillic symbols
    \usepackage[utf8]{inputenc}
    \usepackage[english, russian]{babel}


    
    % Colors for the hyperref package
    \definecolor{urlcolor}{rgb}{0,.145,.698}
    \definecolor{linkcolor}{rgb}{.71,0.21,0.01}
    \definecolor{citecolor}{rgb}{.12,.54,.11}

    % ANSI colors
    \definecolor{ansi-black}{HTML}{3E424D}
    \definecolor{ansi-black-intense}{HTML}{282C36}
    \definecolor{ansi-red}{HTML}{E75C58}
    \definecolor{ansi-red-intense}{HTML}{B22B31}
    \definecolor{ansi-green}{HTML}{00A250}
    \definecolor{ansi-green-intense}{HTML}{007427}
    \definecolor{ansi-yellow}{HTML}{DDB62B}
    \definecolor{ansi-yellow-intense}{HTML}{B27D12}
    \definecolor{ansi-blue}{HTML}{208FFB}
    \definecolor{ansi-blue-intense}{HTML}{0065CA}
    \definecolor{ansi-magenta}{HTML}{D160C4}
    \definecolor{ansi-magenta-intense}{HTML}{A03196}
    \definecolor{ansi-cyan}{HTML}{60C6C8}
    \definecolor{ansi-cyan-intense}{HTML}{258F8F}
    \definecolor{ansi-white}{HTML}{C5C1B4}
    \definecolor{ansi-white-intense}{HTML}{A1A6B2}
    \definecolor{ansi-default-inverse-fg}{HTML}{FFFFFF}
    \definecolor{ansi-default-inverse-bg}{HTML}{000000}

    % common color for the border for error outputs.
    \definecolor{outerrorbackground}{HTML}{FFDFDF}

    % commands and environments needed by pandoc snippets
    % extracted from the output of `pandoc -s`
    \providecommand{\tightlist}{%
      \setlength{\itemsep}{0pt}\setlength{\parskip}{0pt}}
    \DefineVerbatimEnvironment{Highlighting}{Verbatim}{commandchars=\\\{\}}
    % Add ',fontsize=\small' for more characters per line
    \newenvironment{Shaded}{}{}
    \newcommand{\KeywordTok}[1]{\textcolor[rgb]{0.00,0.44,0.13}{\textbf{{#1}}}}
    \newcommand{\DataTypeTok}[1]{\textcolor[rgb]{0.56,0.13,0.00}{{#1}}}
    \newcommand{\DecValTok}[1]{\textcolor[rgb]{0.25,0.63,0.44}{{#1}}}
    \newcommand{\BaseNTok}[1]{\textcolor[rgb]{0.25,0.63,0.44}{{#1}}}
    \newcommand{\FloatTok}[1]{\textcolor[rgb]{0.25,0.63,0.44}{{#1}}}
    \newcommand{\CharTok}[1]{\textcolor[rgb]{0.25,0.44,0.63}{{#1}}}
    \newcommand{\StringTok}[1]{\textcolor[rgb]{0.25,0.44,0.63}{{#1}}}
    \newcommand{\CommentTok}[1]{\textcolor[rgb]{0.38,0.63,0.69}{\textit{{#1}}}}
    \newcommand{\OtherTok}[1]{\textcolor[rgb]{0.00,0.44,0.13}{{#1}}}
    \newcommand{\AlertTok}[1]{\textcolor[rgb]{1.00,0.00,0.00}{\textbf{{#1}}}}
    \newcommand{\FunctionTok}[1]{\textcolor[rgb]{0.02,0.16,0.49}{{#1}}}
    \newcommand{\RegionMarkerTok}[1]{{#1}}
    \newcommand{\ErrorTok}[1]{\textcolor[rgb]{1.00,0.00,0.00}{\textbf{{#1}}}}
    \newcommand{\NormalTok}[1]{{#1}}
    
    % Additional commands for more recent versions of Pandoc
    \newcommand{\ConstantTok}[1]{\textcolor[rgb]{0.53,0.00,0.00}{{#1}}}
    \newcommand{\SpecialCharTok}[1]{\textcolor[rgb]{0.25,0.44,0.63}{{#1}}}
    \newcommand{\VerbatimStringTok}[1]{\textcolor[rgb]{0.25,0.44,0.63}{{#1}}}
    \newcommand{\SpecialStringTok}[1]{\textcolor[rgb]{0.73,0.40,0.53}{{#1}}}
    \newcommand{\ImportTok}[1]{{#1}}
    \newcommand{\DocumentationTok}[1]{\textcolor[rgb]{0.73,0.13,0.13}{\textit{{#1}}}}
    \newcommand{\AnnotationTok}[1]{\textcolor[rgb]{0.38,0.63,0.69}{\textbf{\textit{{#1}}}}}
    \newcommand{\CommentVarTok}[1]{\textcolor[rgb]{0.38,0.63,0.69}{\textbf{\textit{{#1}}}}}
    \newcommand{\VariableTok}[1]{\textcolor[rgb]{0.10,0.09,0.49}{{#1}}}
    \newcommand{\ControlFlowTok}[1]{\textcolor[rgb]{0.00,0.44,0.13}{\textbf{{#1}}}}
    \newcommand{\OperatorTok}[1]{\textcolor[rgb]{0.40,0.40,0.40}{{#1}}}
    \newcommand{\BuiltInTok}[1]{{#1}}
    \newcommand{\ExtensionTok}[1]{{#1}}
    \newcommand{\PreprocessorTok}[1]{\textcolor[rgb]{0.74,0.48,0.00}{{#1}}}
    \newcommand{\AttributeTok}[1]{\textcolor[rgb]{0.49,0.56,0.16}{{#1}}}
    \newcommand{\InformationTok}[1]{\textcolor[rgb]{0.38,0.63,0.69}{\textbf{\textit{{#1}}}}}
    \newcommand{\WarningTok}[1]{\textcolor[rgb]{0.38,0.63,0.69}{\textbf{\textit{{#1}}}}}
    
    
    % Define a nice break command that doesn't care if a line doesn't already
    % exist.
    \def\br{\hspace*{\fill} \\* }
    % Math Jax compatibility definitions
    \def\gt{>}
    \def\lt{<}
    \let\Oldtex\TeX
    \let\Oldlatex\LaTeX
    \renewcommand{\TeX}{\textrm{\Oldtex}}
    \renewcommand{\LaTeX}{\textrm{\Oldlatex}}
    % Document parameters
    % Document title
    \date{}
    \title{Отчет}
    
    
    
    
    
% Pygments definitions
\makeatletter
\def\PY@reset{\let\PY@it=\relax \let\PY@bf=\relax%
    \let\PY@ul=\relax \let\PY@tc=\relax%
    \let\PY@bc=\relax \let\PY@ff=\relax}
\def\PY@tok#1{\csname PY@tok@#1\endcsname}
\def\PY@toks#1+{\ifx\relax#1\empty\else%
    \PY@tok{#1}\expandafter\PY@toks\fi}
\def\PY@do#1{\PY@bc{\PY@tc{\PY@ul{%
    \PY@it{\PY@bf{\PY@ff{#1}}}}}}}
\def\PY#1#2{\PY@reset\PY@toks#1+\relax+\PY@do{#2}}

\expandafter\def\csname PY@tok@w\endcsname{\def\PY@tc##1{\textcolor[rgb]{0.73,0.73,0.73}{##1}}}
\expandafter\def\csname PY@tok@c\endcsname{\let\PY@it=\textit\def\PY@tc##1{\textcolor[rgb]{0.25,0.50,0.50}{##1}}}
\expandafter\def\csname PY@tok@cp\endcsname{\def\PY@tc##1{\textcolor[rgb]{0.74,0.48,0.00}{##1}}}
\expandafter\def\csname PY@tok@k\endcsname{\let\PY@bf=\textbf\def\PY@tc##1{\textcolor[rgb]{0.00,0.50,0.00}{##1}}}
\expandafter\def\csname PY@tok@kp\endcsname{\def\PY@tc##1{\textcolor[rgb]{0.00,0.50,0.00}{##1}}}
\expandafter\def\csname PY@tok@kt\endcsname{\def\PY@tc##1{\textcolor[rgb]{0.69,0.00,0.25}{##1}}}
\expandafter\def\csname PY@tok@o\endcsname{\def\PY@tc##1{\textcolor[rgb]{0.40,0.40,0.40}{##1}}}
\expandafter\def\csname PY@tok@ow\endcsname{\let\PY@bf=\textbf\def\PY@tc##1{\textcolor[rgb]{0.67,0.13,1.00}{##1}}}
\expandafter\def\csname PY@tok@nb\endcsname{\def\PY@tc##1{\textcolor[rgb]{0.00,0.50,0.00}{##1}}}
\expandafter\def\csname PY@tok@nf\endcsname{\def\PY@tc##1{\textcolor[rgb]{0.00,0.00,1.00}{##1}}}
\expandafter\def\csname PY@tok@nc\endcsname{\let\PY@bf=\textbf\def\PY@tc##1{\textcolor[rgb]{0.00,0.00,1.00}{##1}}}
\expandafter\def\csname PY@tok@nn\endcsname{\let\PY@bf=\textbf\def\PY@tc##1{\textcolor[rgb]{0.00,0.00,1.00}{##1}}}
\expandafter\def\csname PY@tok@ne\endcsname{\let\PY@bf=\textbf\def\PY@tc##1{\textcolor[rgb]{0.82,0.25,0.23}{##1}}}
\expandafter\def\csname PY@tok@nv\endcsname{\def\PY@tc##1{\textcolor[rgb]{0.10,0.09,0.49}{##1}}}
\expandafter\def\csname PY@tok@no\endcsname{\def\PY@tc##1{\textcolor[rgb]{0.53,0.00,0.00}{##1}}}
\expandafter\def\csname PY@tok@nl\endcsname{\def\PY@tc##1{\textcolor[rgb]{0.63,0.63,0.00}{##1}}}
\expandafter\def\csname PY@tok@ni\endcsname{\let\PY@bf=\textbf\def\PY@tc##1{\textcolor[rgb]{0.60,0.60,0.60}{##1}}}
\expandafter\def\csname PY@tok@na\endcsname{\def\PY@tc##1{\textcolor[rgb]{0.49,0.56,0.16}{##1}}}
\expandafter\def\csname PY@tok@nt\endcsname{\let\PY@bf=\textbf\def\PY@tc##1{\textcolor[rgb]{0.00,0.50,0.00}{##1}}}
\expandafter\def\csname PY@tok@nd\endcsname{\def\PY@tc##1{\textcolor[rgb]{0.67,0.13,1.00}{##1}}}
\expandafter\def\csname PY@tok@s\endcsname{\def\PY@tc##1{\textcolor[rgb]{0.73,0.13,0.13}{##1}}}
\expandafter\def\csname PY@tok@sd\endcsname{\let\PY@it=\textit\def\PY@tc##1{\textcolor[rgb]{0.73,0.13,0.13}{##1}}}
\expandafter\def\csname PY@tok@si\endcsname{\let\PY@bf=\textbf\def\PY@tc##1{\textcolor[rgb]{0.73,0.40,0.53}{##1}}}
\expandafter\def\csname PY@tok@se\endcsname{\let\PY@bf=\textbf\def\PY@tc##1{\textcolor[rgb]{0.73,0.40,0.13}{##1}}}
\expandafter\def\csname PY@tok@sr\endcsname{\def\PY@tc##1{\textcolor[rgb]{0.73,0.40,0.53}{##1}}}
\expandafter\def\csname PY@tok@ss\endcsname{\def\PY@tc##1{\textcolor[rgb]{0.10,0.09,0.49}{##1}}}
\expandafter\def\csname PY@tok@sx\endcsname{\def\PY@tc##1{\textcolor[rgb]{0.00,0.50,0.00}{##1}}}
\expandafter\def\csname PY@tok@m\endcsname{\def\PY@tc##1{\textcolor[rgb]{0.40,0.40,0.40}{##1}}}
\expandafter\def\csname PY@tok@gh\endcsname{\let\PY@bf=\textbf\def\PY@tc##1{\textcolor[rgb]{0.00,0.00,0.50}{##1}}}
\expandafter\def\csname PY@tok@gu\endcsname{\let\PY@bf=\textbf\def\PY@tc##1{\textcolor[rgb]{0.50,0.00,0.50}{##1}}}
\expandafter\def\csname PY@tok@gd\endcsname{\def\PY@tc##1{\textcolor[rgb]{0.63,0.00,0.00}{##1}}}
\expandafter\def\csname PY@tok@gi\endcsname{\def\PY@tc##1{\textcolor[rgb]{0.00,0.63,0.00}{##1}}}
\expandafter\def\csname PY@tok@gr\endcsname{\def\PY@tc##1{\textcolor[rgb]{1.00,0.00,0.00}{##1}}}
\expandafter\def\csname PY@tok@ge\endcsname{\let\PY@it=\textit}
\expandafter\def\csname PY@tok@gs\endcsname{\let\PY@bf=\textbf}
\expandafter\def\csname PY@tok@gp\endcsname{\let\PY@bf=\textbf\def\PY@tc##1{\textcolor[rgb]{0.00,0.00,0.50}{##1}}}
\expandafter\def\csname PY@tok@go\endcsname{\def\PY@tc##1{\textcolor[rgb]{0.53,0.53,0.53}{##1}}}
\expandafter\def\csname PY@tok@gt\endcsname{\def\PY@tc##1{\textcolor[rgb]{0.00,0.27,0.87}{##1}}}
\expandafter\def\csname PY@tok@err\endcsname{\def\PY@bc##1{\setlength{\fboxsep}{0pt}\fcolorbox[rgb]{1.00,0.00,0.00}{1,1,1}{\strut ##1}}}
\expandafter\def\csname PY@tok@kc\endcsname{\let\PY@bf=\textbf\def\PY@tc##1{\textcolor[rgb]{0.00,0.50,0.00}{##1}}}
\expandafter\def\csname PY@tok@kd\endcsname{\let\PY@bf=\textbf\def\PY@tc##1{\textcolor[rgb]{0.00,0.50,0.00}{##1}}}
\expandafter\def\csname PY@tok@kn\endcsname{\let\PY@bf=\textbf\def\PY@tc##1{\textcolor[rgb]{0.00,0.50,0.00}{##1}}}
\expandafter\def\csname PY@tok@kr\endcsname{\let\PY@bf=\textbf\def\PY@tc##1{\textcolor[rgb]{0.00,0.50,0.00}{##1}}}
\expandafter\def\csname PY@tok@bp\endcsname{\def\PY@tc##1{\textcolor[rgb]{0.00,0.50,0.00}{##1}}}
\expandafter\def\csname PY@tok@fm\endcsname{\def\PY@tc##1{\textcolor[rgb]{0.00,0.00,1.00}{##1}}}
\expandafter\def\csname PY@tok@vc\endcsname{\def\PY@tc##1{\textcolor[rgb]{0.10,0.09,0.49}{##1}}}
\expandafter\def\csname PY@tok@vg\endcsname{\def\PY@tc##1{\textcolor[rgb]{0.10,0.09,0.49}{##1}}}
\expandafter\def\csname PY@tok@vi\endcsname{\def\PY@tc##1{\textcolor[rgb]{0.10,0.09,0.49}{##1}}}
\expandafter\def\csname PY@tok@vm\endcsname{\def\PY@tc##1{\textcolor[rgb]{0.10,0.09,0.49}{##1}}}
\expandafter\def\csname PY@tok@sa\endcsname{\def\PY@tc##1{\textcolor[rgb]{0.73,0.13,0.13}{##1}}}
\expandafter\def\csname PY@tok@sb\endcsname{\def\PY@tc##1{\textcolor[rgb]{0.73,0.13,0.13}{##1}}}
\expandafter\def\csname PY@tok@sc\endcsname{\def\PY@tc##1{\textcolor[rgb]{0.73,0.13,0.13}{##1}}}
\expandafter\def\csname PY@tok@dl\endcsname{\def\PY@tc##1{\textcolor[rgb]{0.73,0.13,0.13}{##1}}}
\expandafter\def\csname PY@tok@s2\endcsname{\def\PY@tc##1{\textcolor[rgb]{0.73,0.13,0.13}{##1}}}
\expandafter\def\csname PY@tok@sh\endcsname{\def\PY@tc##1{\textcolor[rgb]{0.73,0.13,0.13}{##1}}}
\expandafter\def\csname PY@tok@s1\endcsname{\def\PY@tc##1{\textcolor[rgb]{0.73,0.13,0.13}{##1}}}
\expandafter\def\csname PY@tok@mb\endcsname{\def\PY@tc##1{\textcolor[rgb]{0.40,0.40,0.40}{##1}}}
\expandafter\def\csname PY@tok@mf\endcsname{\def\PY@tc##1{\textcolor[rgb]{0.40,0.40,0.40}{##1}}}
\expandafter\def\csname PY@tok@mh\endcsname{\def\PY@tc##1{\textcolor[rgb]{0.40,0.40,0.40}{##1}}}
\expandafter\def\csname PY@tok@mi\endcsname{\def\PY@tc##1{\textcolor[rgb]{0.40,0.40,0.40}{##1}}}
\expandafter\def\csname PY@tok@il\endcsname{\def\PY@tc##1{\textcolor[rgb]{0.40,0.40,0.40}{##1}}}
\expandafter\def\csname PY@tok@mo\endcsname{\def\PY@tc##1{\textcolor[rgb]{0.40,0.40,0.40}{##1}}}
\expandafter\def\csname PY@tok@ch\endcsname{\let\PY@it=\textit\def\PY@tc##1{\textcolor[rgb]{0.25,0.50,0.50}{##1}}}
\expandafter\def\csname PY@tok@cm\endcsname{\let\PY@it=\textit\def\PY@tc##1{\textcolor[rgb]{0.25,0.50,0.50}{##1}}}
\expandafter\def\csname PY@tok@cpf\endcsname{\let\PY@it=\textit\def\PY@tc##1{\textcolor[rgb]{0.25,0.50,0.50}{##1}}}
\expandafter\def\csname PY@tok@c1\endcsname{\let\PY@it=\textit\def\PY@tc##1{\textcolor[rgb]{0.25,0.50,0.50}{##1}}}
\expandafter\def\csname PY@tok@cs\endcsname{\let\PY@it=\textit\def\PY@tc##1{\textcolor[rgb]{0.25,0.50,0.50}{##1}}}

\def\PYZbs{\char`\\}
\def\PYZus{\char`\_}
\def\PYZob{\char`\{}
\def\PYZcb{\char`\}}
\def\PYZca{\char`\^}
\def\PYZam{\char`\&}
\def\PYZlt{\char`\<}
\def\PYZgt{\char`\>}
\def\PYZsh{\char`\#}
\def\PYZpc{\char`\%}
\def\PYZdl{\char`\$}
\def\PYZhy{\char`\-}
\def\PYZsq{\char`\'}
\def\PYZdq{\char`\"}
\def\PYZti{\char`\~}
% for compatibility with earlier versions
\def\PYZat{@}
\def\PYZlb{[}
\def\PYZrb{]}
\makeatother


    % For linebreaks inside Verbatim environment from package fancyvrb. 
    \makeatletter
        \newbox\Wrappedcontinuationbox 
        \newbox\Wrappedvisiblespacebox 
        \newcommand*\Wrappedvisiblespace {\textcolor{red}{\textvisiblespace}} 
        \newcommand*\Wrappedcontinuationsymbol {\textcolor{red}{\llap{\tiny$\m@th\hookrightarrow$}}} 
        \newcommand*\Wrappedcontinuationindent {3ex } 
        \newcommand*\Wrappedafterbreak {\kern\Wrappedcontinuationindent\copy\Wrappedcontinuationbox} 
        % Take advantage of the already applied Pygments mark-up to insert 
        % potential linebreaks for TeX processing. 
        %        {, <, #, %, $, ' and ": go to next line. 
        %        _, }, ^, &, >, - and ~: stay at end of broken line. 
        % Use of \textquotesingle for straight quote. 
        \newcommand*\Wrappedbreaksatspecials {% 
            \def\PYGZus{\discretionary{\char`\_}{\Wrappedafterbreak}{\char`\_}}% 
            \def\PYGZob{\discretionary{}{\Wrappedafterbreak\char`\{}{\char`\{}}% 
            \def\PYGZcb{\discretionary{\char`\}}{\Wrappedafterbreak}{\char`\}}}% 
            \def\PYGZca{\discretionary{\char`\^}{\Wrappedafterbreak}{\char`\^}}% 
            \def\PYGZam{\discretionary{\char`\&}{\Wrappedafterbreak}{\char`\&}}% 
            \def\PYGZlt{\discretionary{}{\Wrappedafterbreak\char`\<}{\char`\<}}% 
            \def\PYGZgt{\discretionary{\char`\>}{\Wrappedafterbreak}{\char`\>}}% 
            \def\PYGZsh{\discretionary{}{\Wrappedafterbreak\char`\#}{\char`\#}}% 
            \def\PYGZpc{\discretionary{}{\Wrappedafterbreak\char`\%}{\char`\%}}% 
            \def\PYGZdl{\discretionary{}{\Wrappedafterbreak\char`\$}{\char`\$}}% 
            \def\PYGZhy{\discretionary{\char`\-}{\Wrappedafterbreak}{\char`\-}}% 
            \def\PYGZsq{\discretionary{}{\Wrappedafterbreak\textquotesingle}{\textquotesingle}}% 
            \def\PYGZdq{\discretionary{}{\Wrappedafterbreak\char`\"}{\char`\"}}% 
            \def\PYGZti{\discretionary{\char`\~}{\Wrappedafterbreak}{\char`\~}}% 
        } 
        % Some characters . , ; ? ! / are not pygmentized. 
        % This macro makes them "active" and they will insert potential linebreaks 
        \newcommand*\Wrappedbreaksatpunct {% 
            \lccode`\~`\.\lowercase{\def~}{\discretionary{\hbox{\char`\.}}{\Wrappedafterbreak}{\hbox{\char`\.}}}% 
            \lccode`\~`\,\lowercase{\def~}{\discretionary{\hbox{\char`\,}}{\Wrappedafterbreak}{\hbox{\char`\,}}}% 
            \lccode`\~`\;\lowercase{\def~}{\discretionary{\hbox{\char`\;}}{\Wrappedafterbreak}{\hbox{\char`\;}}}% 
            \lccode`\~`\:\lowercase{\def~}{\discretionary{\hbox{\char`\:}}{\Wrappedafterbreak}{\hbox{\char`\:}}}% 
            \lccode`\~`\?\lowercase{\def~}{\discretionary{\hbox{\char`\?}}{\Wrappedafterbreak}{\hbox{\char`\?}}}% 
            \lccode`\~`\!\lowercase{\def~}{\discretionary{\hbox{\char`\!}}{\Wrappedafterbreak}{\hbox{\char`\!}}}% 
            \lccode`\~`\/\lowercase{\def~}{\discretionary{\hbox{\char`\/}}{\Wrappedafterbreak}{\hbox{\char`\/}}}% 
            \catcode`\.\active
            \catcode`\,\active 
            \catcode`\;\active
            \catcode`\:\active
            \catcode`\?\active
            \catcode`\!\active
            \catcode`\/\active 
            \lccode`\~`\~ 	
        }
    \makeatother

    \let\OriginalVerbatim=\Verbatim
    \makeatletter
    \renewcommand{\Verbatim}[1][1]{%
        %\parskip\z@skip
        \sbox\Wrappedcontinuationbox {\Wrappedcontinuationsymbol}%
        \sbox\Wrappedvisiblespacebox {\FV@SetupFont\Wrappedvisiblespace}%
        \def\FancyVerbFormatLine ##1{\hsize\linewidth
            \vtop{\raggedright\hyphenpenalty\z@\exhyphenpenalty\z@
                \doublehyphendemerits\z@\finalhyphendemerits\z@
                \strut ##1\strut}%
        }%
        % If the linebreak is at a space, the latter will be displayed as visible
        % space at end of first line, and a continuation symbol starts next line.
        % Stretch/shrink are however usually zero for typewriter font.
        \def\FV@Space {%
            \nobreak\hskip\z@ plus\fontdimen3\font minus\fontdimen4\font
            \discretionary{\copy\Wrappedvisiblespacebox}{\Wrappedafterbreak}
            {\kern\fontdimen2\font}%
        }%
        
        % Allow breaks at special characters using \PYG... macros.
        \Wrappedbreaksatspecials
        % Breaks at punctuation characters . , ; ? ! and / need catcode=\active 	
        \OriginalVerbatim[#1,codes*=\Wrappedbreaksatpunct]%
    }
    \makeatother

    % Exact colors from NB
    \definecolor{incolor}{HTML}{303F9F}
    \definecolor{outcolor}{HTML}{D84315}
    \definecolor{cellborder}{HTML}{CFCFCF}
    \definecolor{cellbackground}{HTML}{F7F7F7}
    
    % prompt
    \makeatletter
    \newcommand{\boxspacing}{\kern\kvtcb@left@rule\kern\kvtcb@boxsep}
    \makeatother
    \newcommand{\prompt}[4]{
        {\ttfamily\llap{{\color{#2}[#3]:\hspace{3pt}#4}}\vspace{-\baselineskip}}
    }
    

    
    % Prevent overflowing lines due to hard-to-break entities
    \sloppy 
    % Setup hyperref package
    \hypersetup{
      breaklinks=true,  % so long urls are correctly broken across lines
      colorlinks=true,
      urlcolor=urlcolor,
      linkcolor=linkcolor,
      citecolor=citecolor,
      }
    % Slightly bigger margins than the latex defaults
    
    \geometry{verbose,tmargin=1in,bmargin=1in,lmargin=1.5cm,rmargin=1cm}
    
    

\begin{document}
    
    \maketitle
    
    

    
    \hypertarget{ux430ux432ux442ux43eux440ux44b}{%
\section{Авторы}\label{ux430ux432ux442ux43eux440ux44b}}

Студенты группы M3439:

\begin{itemize}
\tightlist
\item
  Тепляков Валерий
\item
  Плешаков Алексей
\item
  Филипчик Андрей
\end{itemize}

    \hypertarget{source-code}{%
\section{Source code}\label{source-code}}

Исходный код можно посмотреть
\href{https://github.com/Mervap/OptimizationMethods/tree/master/lab1}{тут}

    \hypertarget{ux437ux430ux434ux430ux43dux438ux435-1}{%
\section{Задание 1}\label{ux437ux430ux434ux430ux43dux438ux435-1}}

    Для сравнения методов одномерного поиска возьмем следующую унимодальную
на отрезке {[}2.2, 2.8{]} функцию: \[f(x) = |\sin(x^2)|\]

    \begin{center}
    \adjustimage{max size={0.9\linewidth}{0.9\paperheight}}{report_files/report_5_0.png}
    \end{center}
    { \hspace*{\fill} \\}
    
    Сравним метод дихотомии, метод золотого сечения и метод Фибоначчи по
количеству итераций и количеству вычислений функции в зависимости от
запрашиваемой точности.

На графиках видно, что метод золотого сечения и метод Фибоначчи почти не
отличаются по данным характеристикам. Метод Фибонначи на данной функции
показывает результат всегда не хуже метода золотого сечения.

Метод дихотомии же использует меньшее количество итераций, но требует в
\(2\) раза больше вычислений функции. Это ожидаемый результат, так как
другие два метода переиспользуют уже вычисленные значения.

    \begin{center}
    \adjustimage{max size={0.9\linewidth}{0.9\paperheight}}{report_files/report_7_0.png}
    \end{center}
    { \hspace*{\fill} \\}
    
    Также посмотрим как изменяется отрезок при переходе к следующей итерации
(правый график приближение левого). Видно, что на первых итерациях метод
дихотомии сокращает интервал сильнее и быстро сходится. Другие 2 метода
идут примерно наравне, лишь на большом масштабе видна разница (возможно
связанная просто с погрешностями вычислений)

    \begin{center}
    \adjustimage{max size={0.9\linewidth}{0.9\paperheight}}{report_files/report_9_0.png}
    \end{center}
    { \hspace*{\fill} \\}
    
    \newpage

\hypertarget{ux437ux430ux434ux430ux43dux438ux435-2}{%
\section{Задание 2}\label{ux437ux430ux434ux430ux43dux438ux435-2}}

    \begin{Verbatim}[commandchars=\\\{\}]
0 [0.5, 0.1] Golden ratio
0 [0.5, 0.1] Fibonacci
0 [0.5, 0.1] Dichotomy
0 [0.5, 0.1] Binary search
0 [0, 1] Golden ratio
0 [0, 1] Fibonacci
0 [0, 1] Dichotomy
0 [0, 1] Binary search
0 [-3, 0.4] Golden ratio
0 [-3, 0.4] Fibonacci
0 [-3, 0.4] Dichotomy
0 [-3, 0.4] Binary search
0 [-1, 1] Golden ratio
0 [-1, 1] Fibonacci
0 [-1, 1] Dichotomy
0 [-1, 1] Binary search
1 [0.5, 0.1] Golden ratio
1 [0.5, 0.1] Fibonacci
1 [0.5, 0.1] Dichotomy
1 [0.5, 0.1] Binary search
1 [0, 1] Golden ratio
1 [0, 1] Fibonacci
1 [0, 1] Dichotomy
1 [0, 1] Binary search
1 [-3, 0.4] Golden ratio
1 [-3, 0.4] Fibonacci
1 [-3, 0.4] Dichotomy
1 [-3, 0.4] Binary search
1 [-1, 1] Golden ratio
1 [-1, 1] Fibonacci
1 [-1, 1] Dichotomy
1 [-1, 1] Binary search
2 [0.5, 0.1] Golden ratio
2 [0.5, 0.1] Fibonacci
2 [0.5, 0.1] Dichotomy
2 [0.5, 0.1] Binary search
2 [0, 1] Golden ratio
2 [0, 1] Fibonacci
2 [0, 1] Dichotomy
2 [0, 1] Binary search
2 [-3, 0.4] Golden ratio
2 [-3, 0.4] Fibonacci
2 [-3, 0.4] Dichotomy
2 [-3, 0.4] Binary search
2 [-1, 1] Golden ratio
2 [-1, 1] Fibonacci
2 [-1, 1] Dichotomy
2 [-1, 1] Binary search
    \end{Verbatim}

    Используется наискорейший градиентный спуск. В качестве процедур
одномерной оптимизации используются методы из задания 1 + метод средней
точки (он же просто бинарный поиск) на основе производной функции

Для экспериментов использовались следующие функции:

\begin{enumerate}
\def\labelenumi{\arabic{enumi}.}
\setcounter{enumi}{-1}
\tightlist
\item
  \(f(x, y) = 3x^2+7y^2+2yx-x\)
\item
  \(f(x, y) = (14-x)^2+88(y - 4x + 7)^2\)
\item
  \(f(x, y) = xe^{-x^2-y^2}\)
\end{enumerate}

Метод запускался из следующих начальных точек:

\begin{enumerate}
\def\labelenumi{\arabic{enumi}.}
\setcounter{enumi}{-1}
\tightlist
\item
  (0.5, 0.1)
\item
  (0, 1)
\item
  (-3, 0.4)
\item
  (-1, 1)
\end{enumerate}

Подробные результаты запусков можно изучить в таблице:
 
            
\prompt{Out}{outcolor}{7}{}
    
    \begin{tabular}{rlllrrrl}
\toprule
 Function &      Optimizer & Start point &              Result point &  Iterations &  Function calls &  Gradient calls & Elapsed time \\
\midrule
        0 &   Golden ratio &  [0.5, 0.1] &   (0.1686819, -0.0120458) &          12 &             455 &              13 &         0.5s \\
        0 &      Fibonacci &  [0.5, 0.1] &   (0.1686819, -0.0120458) &          12 &             455 &              13 &         0.6s \\
        0 &      Dichotomy &  [0.5, 0.1] &   (0.1686818, -0.0120457) &          12 &             650 &              13 &         0.6s \\
        0 &  Binary search &  [0.5, 0.1] &   (0.1686819, -0.0120458) &          12 &               0 &             338 &         0.6s \\
        0 &   Golden ratio &      [0, 1] &   (0.1686744, -0.0120465) &           6 &             245 &               7 &         0.3s \\
        0 &      Fibonacci &      [0, 1] &   (0.1686744, -0.0120465) &           6 &             245 &               7 &         0.6s \\
        0 &      Dichotomy &      [0, 1] &   (0.1686745, -0.0120465) &           6 &             350 &               7 &         0.3s \\
        0 &  Binary search &      [0, 1] &   (0.1686744, -0.0120465) &           6 &               0 &             182 &         0.3s \\
        0 &   Golden ratio &   [-3, 0.4] &   (0.1686736, -0.0120548) &           3 &             140 &               4 &         0.1s \\
        0 &      Fibonacci &   [-3, 0.4] &   (0.1686736, -0.0120548) &           3 &             140 &               4 &         0.1s \\
        0 &      Dichotomy &   [-3, 0.4] &   (0.1686736, -0.0120548) &           3 &             200 &               4 &         0.1s \\
        0 &  Binary search &   [-3, 0.4] &   (0.1686736, -0.0120548) &           3 &               0 &             104 &         0.2s \\
        0 &   Golden ratio &     [-1, 1] &   (0.1686655, -0.0120402) &          12 &             455 &              13 &         0.6s \\
        0 &      Fibonacci &     [-1, 1] &   (0.1686655, -0.0120402) &          12 &             455 &              13 &         0.6s \\
        0 &      Dichotomy &     [-1, 1] &   (0.1686656, -0.0120402) &          12 &             650 &              13 &         0.7s \\
        0 &  Binary search &     [-1, 1] &   (0.1686655, -0.0120402) &          12 &               0 &             338 &         0.5s \\
        1 &   Golden ratio &  [0.5, 0.1] &  (13.9988590, 48.9953925) &         174 &            6125 &             175 &        0.60s \\
        1 &      Fibonacci &  [0.5, 0.1] &  (13.9937287, 48.9748573) &        3278 &          114765 &            3279 &       1.348s \\
        1 &      Dichotomy &  [0.5, 0.1] &  (13.9932726, 48.9730319) &        4558 &          227950 &            4559 &       1.759s \\
        1 &  Binary search &  [0.5, 0.1] &  (13.9977413, 48.9909183) &         606 &               0 &           15782 &       0.276s \\
        1 &   Golden ratio &      [0, 1] &  (13.9982797, 48.9930759) &         402 &           14105 &             403 &       0.116s \\
        1 &      Fibonacci &      [0, 1] &  (13.9975424, 48.9901220) &         692 &           24255 &             693 &       0.246s \\
        1 &      Dichotomy &      [0, 1] &  (13.9957280, 48.9828597) &        1944 &           97250 &            1945 &       0.755s \\
        1 &  Binary search &      [0, 1] &  (13.9963823, 48.9854789) &        1448 &               0 &           37674 &       0.587s \\
        1 &   Golden ratio &   [-3, 0.4] &  (13.9956479, 48.9825392) &        2048 &           71715 &            2049 &       0.631s \\
        1 &      Fibonacci &   [-3, 0.4] &  (13.9986909, 48.9947223) &         234 &            8225 &             235 &        0.82s \\
        1 &      Dichotomy &   [-3, 0.4] &  (13.9982071, 48.9927832) &         404 &           20250 &             405 &       0.161s \\
        1 &  Binary search &   [-3, 0.4] &  (13.9916443, 48.9665149) &        6948 &               0 &          180674 &       2.981s \\
        1 &   Golden ratio &     [-1, 1] &  (13.9971676, 48.9886222) &         900 &           31535 &             901 &       0.276s \\
        1 &      Fibonacci &     [-1, 1] &  (13.9983873, 48.9935053) &         340 &           11935 &             341 &       0.114s \\
        1 &      Dichotomy &     [-1, 1] &  (13.9971682, 48.9886251) &         940 &           47050 &             941 &       0.344s \\
        1 &  Binary search &     [-1, 1] &  (13.9946088, 48.9783802) &        3024 &               0 &           78650 &       1.200s \\
        2 &   Golden ratio &  [0.5, 0.1] &   (-0.7071046, 0.0000059) &          10 &             385 &              11 &         0.3s \\
        2 &      Fibonacci &  [0.5, 0.1] &   (-0.7071046, 0.0000059) &          10 &             385 &              11 &         0.3s \\
        2 &      Dichotomy &  [0.5, 0.1] &   (-0.7071085, 0.0000078) &          10 &             550 &              11 &         0.3s \\
        2 &  Binary search &  [0.5, 0.1] &   (-0.7071046, 0.0000059) &          10 &               0 &             286 &         0.5s \\
        2 &   Golden ratio &      [0, 1] &   (-0.7071090, 0.0000027) &          11 &             420 &              12 &         0.4s \\
        2 &      Fibonacci &      [0, 1] &   (-0.7071090, 0.0000027) &          11 &             420 &              12 &         0.5s \\
        2 &      Dichotomy &      [0, 1] &   (-0.7071113, 0.0000025) &          11 &             600 &              12 &         0.5s \\
        2 &  Binary search &      [0, 1] &   (-0.7071090, 0.0000027) &          11 &               0 &             312 &         0.6s \\
        2 &   Golden ratio &   [-3, 0.4] &   (-0.7071040, 0.0000075) &         126 &            4445 &             127 &        0.38s \\
        2 &      Fibonacci &   [-3, 0.4] &   (-0.7071040, 0.0000075) &         126 &            4445 &             127 &        0.43s \\
        2 &      Dichotomy &   [-3, 0.4] &   (-0.7071028, 0.0000066) &         126 &            6350 &             127 &        0.46s \\
        2 &  Binary search &   [-3, 0.4] &   (-0.7071040, 0.0000075) &         126 &               0 &            3302 &        0.65s \\
        2 &   Golden ratio &     [-1, 1] &   (-0.7071035, 0.0000059) &          11 &             420 &              12 &         0.4s \\
        2 &      Fibonacci &     [-1, 1] &   (-0.7071034, 0.0000059) &          11 &             420 &              12 &         0.5s \\
        2 &      Dichotomy &     [-1, 1] &   (-0.7071101, 0.0000096) &          11 &             600 &              12 &         0.7s \\
        2 &  Binary search &     [-1, 1] &   (-0.7071035, 0.0000059) &          11 &               0 &             312 &         0.9s \\
\bottomrule
\end{tabular}


    

    Сравнение скорости сходимости проводилось по затраченному процессорному
времени, так как количество итераций и количество вычислений функции не
очень репрентативная иноформация. Нас все равно интересует как быстро мы
получим численный ответ в реальной жизни.

Агрегированные результаты можно видеть ниже:

    \begin{center}
    \adjustimage{max size={0.9\linewidth}{0.9\paperheight}}{report_files/report_16_0.png}
    \end{center}
    { \hspace*{\fill} \\}
    
    Видно, что на 0 и 2 функции методы ведут себя сравнимо. Бинарный поиск
проигрывает другим методам скорее всего из-за не самой эффективной
реализации вычисления градиента, который он много раз вычисляет

Очень велика разница между методом Фибоначчи и остальными методами на 1
функции. Вероятнее всего это связанно с тем, что данная функция плохо
обусловленна (будет видно далее по линиям уровня), из-за чего
градиентный спуск в целом сходится плохо. А также из-за того, что из-за
операций над большими числами алгоритму не хватает точности на каждом
вычислении шага.

    \newpage

\hypertarget{ux437ux430ux434ux430ux43dux438ux435-3}{%
\section{Задание 3}\label{ux437ux430ux434ux430ux43dux438ux435-3}}

На графиках ниже можно наблюдать траектории градиентного спуска (точка
на графике - стартовое значение, крестик - результат алгоритма)

Также на графиках изображены линии уровня и сетка раскрашена по
значениям функции в точках (чем темнее, тем меньше значение)

Правые графики являются приближением левых с концентрацией на последних
итерациях

    \begin{center}
    \adjustimage{max size={0.9\linewidth}{0.9\paperheight}}{report_files/report_19_0.png}
    \end{center}
    { \hspace*{\fill} \\}
    
    \newpage

\hypertarget{ux437ux430ux434ux430ux43dux438ux435-4}{%
\section{Задание 4}\label{ux437ux430ux434ux430ux43dux438ux435-4}}

    \begin{Verbatim}[commandchars=\\\{\}]
n = 3; k = 1.0
n = 3; k = 35.44827586206897
n = 3; k = 69.89655172413794
n = 3; k = 104.3448275862069
n = 3; k = 138.79310344827587
n = 3; k = 173.24137931034483
n = 3; k = 207.6896551724138
n = 3; k = 242.1379310344828
n = 3; k = 276.58620689655174
n = 3; k = 311.0344827586207
n = 3; k = 345.48275862068965
n = 3; k = 379.93103448275866
n = 3; k = 414.3793103448276
n = 3; k = 448.82758620689657
n = 3; k = 483.2758620689656
n = 3; k = 517.7241379310345
n = 3; k = 552.1724137931035
n = 3; k = 586.6206896551724
n = 3; k = 621.0689655172414
n = 3; k = 655.5172413793103
n = 3; k = 689.9655172413793
n = 3; k = 724.4137931034484
n = 3; k = 758.8620689655173
n = 3; k = 793.3103448275863
n = 3; k = 827.7586206896552
n = 3; k = 862.2068965517242
n = 3; k = 896.6551724137931
n = 3; k = 931.1034482758621
n = 3; k = 965.5517241379312
n = 3; k = 1000.0
n = 10; k = 1.0
n = 10; k = 35.44827586206897
n = 10; k = 69.89655172413794
n = 10; k = 104.3448275862069
n = 10; k = 138.79310344827587
n = 10; k = 173.24137931034483
n = 10; k = 207.6896551724138
n = 10; k = 242.1379310344828
n = 10; k = 276.58620689655174
n = 10; k = 311.0344827586207
n = 10; k = 345.48275862068965
n = 10; k = 379.93103448275866
n = 10; k = 414.3793103448276
n = 10; k = 448.82758620689657
n = 10; k = 483.2758620689656
n = 10; k = 517.7241379310345
n = 10; k = 552.1724137931035
n = 10; k = 586.6206896551724
n = 10; k = 621.0689655172414
n = 10; k = 655.5172413793103
n = 10; k = 689.9655172413793
n = 10; k = 724.4137931034484
n = 10; k = 758.8620689655173
n = 10; k = 793.3103448275863
n = 10; k = 827.7586206896552
n = 10; k = 862.2068965517242
n = 10; k = 896.6551724137931
n = 10; k = 931.1034482758621
n = 10; k = 965.5517241379312
n = 10; k = 1000.0
n = 100; k = 1.0
n = 100; k = 35.44827586206897
n = 100; k = 69.89655172413794
n = 100; k = 104.3448275862069
n = 100; k = 138.79310344827587
n = 100; k = 173.24137931034483
n = 100; k = 207.6896551724138
n = 100; k = 242.1379310344828
n = 100; k = 276.58620689655174
n = 100; k = 311.0344827586207
n = 100; k = 345.48275862068965
n = 100; k = 379.93103448275866
n = 100; k = 414.3793103448276
n = 100; k = 448.82758620689657
n = 100; k = 483.2758620689656
n = 100; k = 517.7241379310345
n = 100; k = 552.1724137931035
n = 100; k = 586.6206896551724
n = 100; k = 621.0689655172414
n = 100; k = 655.5172413793103
n = 100; k = 689.9655172413793
n = 100; k = 724.4137931034484
n = 100; k = 758.8620689655173
n = 100; k = 793.3103448275863
n = 100; k = 827.7586206896552
n = 100; k = 862.2068965517242
n = 100; k = 896.6551724137931
n = 100; k = 931.1034482758621
n = 100; k = 965.5517241379312
n = 100; k = 1000.0
n = 1000; k = 1.0
n = 1000; k = 35.44827586206897
n = 1000; k = 69.89655172413794
n = 1000; k = 104.3448275862069
n = 1000; k = 138.79310344827587
n = 1000; k = 173.24137931034483
n = 1000; k = 207.6896551724138
n = 1000; k = 242.1379310344828
n = 1000; k = 276.58620689655174
n = 1000; k = 311.0344827586207
n = 1000; k = 345.48275862068965
n = 1000; k = 379.93103448275866
n = 1000; k = 414.3793103448276
n = 1000; k = 448.82758620689657
n = 1000; k = 483.2758620689656
n = 1000; k = 517.7241379310345
n = 1000; k = 552.1724137931035
n = 1000; k = 586.6206896551724
n = 1000; k = 621.0689655172414
n = 1000; k = 655.5172413793103
n = 1000; k = 689.9655172413793
n = 1000; k = 724.4137931034484
n = 1000; k = 758.8620689655173
n = 1000; k = 793.3103448275863
n = 1000; k = 827.7586206896552
n = 1000; k = 862.2068965517242
n = 1000; k = 896.6551724137931
n = 1000; k = 931.1034482758621
n = 1000; k = 965.5517241379312
n = 1000; k = 1000.0
    \end{Verbatim}

    Будем генерировать квадратичную задачу следующим образом:

\begin{enumerate}
\def\labelenumi{\arabic{enumi}.}
\tightlist
\item
  Возьмем произвольную матрицу \(Q^`_{[n \times n]}\)
\item
  Произведем над ней сингулярное разложение \(Q^` = U S^` V^T\)
\item
  Пусть \(s_{min}\) --- минимальное сингулярное число, а \(s_{max}\) ---
  максимальное. Отмасштабируем диагональные элементы матрицы
  \(S^` -> S\): \([s_{min}, s_{max}] -> [1, \sqrt{k}]\)
\item
  Посчитаем матрицу \(Q_{1} = U S V^T\)
\item
  Построим матрицу \(Q = Q_1Q_1^T\) обладает числом обусловленности
  \(k\), так как ее минимальное собственное число равно \(1\), а
  максимальное --- \(k\)
\item
  Сгенерируем случайный вектор \(b\)
\item
  Задача \(f(x) = (Qx, x) + (b, x)\) является искомой и обладает числом
  обусловленности \(k\), так как матрица \(Q\) является гессианом и ее
  минимальное собственное число равно 1, а максимальное --- \(k\)
\end{enumerate}

Посчитаем количество итераций в зависимости от числа обусловленности при
\(n \in \{3, 10, 100, 1000\}\):

    На графиках виден тренд, что при увелечении числа обусловленности и
размерности множества число итераций градиентного спуска также растет


    % Add a bibliography block to the postdoc
    
    
    
\end{document}
